\section{参考文献と参照}
参照命令に応じて拡張子が bib であるファイルから情報が読み取られ、適切な番号が割り振られ、その番号順に参考文献が作成される。
例えば、「湖上ら\cite{Kogami2009}は、人間共生型ロボットの感性出力に関する研究を行った。」という文章において、参照の命令は「\verb+\cite{Kogami2009}+」の様に記載されている。
これで参考文献の出力順に応じた番号が自動的に割り振られ、参考文献のページに適切なフォーマットにて出力がなされる。

この引用のラベルは、myrefs.bib ファイルにて、以下の様に記載されている。
\begin{breakbox}
{\small
\begin{verbatim}
@article{Kogami2009,
 author = "湖上 潤 and 宮治 裕 and 富山 健",
 title = "人間共生ロボットにおける擬似感性システムの構築と評価",
 journal = "日本感性工学会論文誌",
 pages = "601-609",
 month = sep,
 year = 2009,
}
\end{verbatim}
}
\end{breakbox}
この \BibTeX の書式は、全て各自で記述しても構わないが、一般的な論文をダウンロードするサイトにおいて出力することができるようになっており、それを利用して良い。

上記例は論文に関する情報であるが、書籍(の一部)\cite{WelfareJapan}、書籍\cite{Nakata2010}、予稿集\cite{Miyaji2003ROMAN}、その他(Webサイトなど)\cite{HTUlatex}で参考文献欄に載せる情報は異なる。
それぞれの書式を記載しておいたので、各自で myrefs.bib ファイルを参照して欲しい。
