\section{対話よるムード推定に基づく楽曲推薦エージェントの開発}
角田ら(2018)[4]は印象をPlutchikの感情モデル[5]の基本8感情を用いて推定した.基本8感情は喜び,信頼,心配,驚き,悲しみ,嫌悪,怒り,予測である.歌詞情報をMeCabによる形態素解析をして単語を抽出する.
その後Word2Vecを利用して,抽出した単語のベクトル変換を行ない,基本8感情との類似度を算出して,一番類似度が高い感情をテキストから得られる印象とした.この研究では喜び・悲しみなどの推定できた印象と嫌悪・怒りなどの推定できなかった印象が存在した.