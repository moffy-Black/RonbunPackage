近年スマートフォンとオンライン音楽配信サービスの発達により,人々は膨大な数の楽曲を手軽に聴けるようになった.
人々がその膨大な楽曲の中から自分の嗜好にあった楽曲を見つけ出すことは困難であるため,楽曲推薦サービスの需要が高まっている.

楽曲推薦サービスの中でも人間の気分や状況に基づいて楽曲を推薦するサービスをコンテキストアウェアシステムと呼ぶ.
そのシステムの中でも日本語歌詞から人間が感じる印象を推定し,ユーザの気分に合わせて楽曲を推薦する研究は未だ発展途上であった.

本研究ではPLSAによって歌詞中に出現する単語を分析することで,4クラスの潜在的トピックにクラスタリングする.そして得られたモデルパラメータをMAP推定することで,潜在的トピックを喜怒哀楽の4つの印象に制限し,歌詞中に出現する単語から感じとれる印象を推定する.
印象は感性平面上にプロットすることで表現する.歌詞をフレーズごとに分割して,フレーズに登場する単語の印象を合計して正規化することでフレーズの印象とする.
フレーズの印象を合計して正規化することで歌詞全体の印象とする.

歌詞とフレーズに対して推定された印象が妥当であるか有効性を示すために実験をおこなった.
その結果喜の印象推定は妥当であることが確認できた.しかし,その他の印象における推定結果は妥当であるとは言えなかった.
喜のクラスだけが印象推定で有効性を示せた理由はMAP推定に用いた事前知識の数が十分にあったことが一番大きな要因として考えられる.
事前知識の数が少ないとMAP推定する際に無情報事前分布を与えるため,MAP推定前の潜在トピックに属するモデルパラメータと変わらない.
そのためうまく印象を推定できなかったと考慮する.
したがって事前知識を十分確保できた場合,本論で設計した印象推定手法が有効であると期待される.