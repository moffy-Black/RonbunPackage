\section{テスト}
この文書は, RedPen の校正をテストするために作られた.
また,それと同時に,どのようにチェックが行われるのかのデモンストレーションをするために利用するファイルである.
まずは,一つの文章(センテンス)に,どれぐらいの文字を書くことができるかについてチェックする.

二つの側面からエラーが出力される.
一つの文章に全角で100字以上の文字が配置されると,理解困難な文章である可能性が高いためセンテンス長エラーになり,同様の理由から50以上の単語で構成されるときには語数エラーとなる.

数がかわるものは洋数字を用い,変わらないものは漢数字を用いるのが一般的である.
1点目や2点目のように示す場合は,洋数字を用いる.
百発百中や四季折々のようなものは漢数字を用いる.
ただし,一つや二つの表現は,単漢字にそろえることとする.
「つ」を使い数える場合には,漢数字を使わないと違和感がある.
従って,この行の1つという表現はエラーとなる.

これかから多くの作業をおこなう必要がある.

以下,いくつかのエラーパターンを提示する.
句読点として、全角の「,」と「.」を使わない場合はエラーになる。
今回の設定では,この文章のように4個を超えて読点が使われるとエラーを表示するように設定をしている.

助詞が連続している場合にはエラーを出す.
たとえば,本屋の棚の上の猫はエラーとなる.

音節が三音節以上であるカタカナ語の場合,最後が長音記号で終わる語は,その長音記号を削除する決まりがある.
たとえば,コンピューターはエラーになる.
スキーは,エラーにならない.

TeX で文章を書いているとき,空白行があるとパラグラフが変わる.
節の中にパラグラフの数が 9個以上になると大きすぎるのでエラーを出力する.

この文章は,二重否定の文章といえなくはない.

費やしては送り仮名が合っている.
しかし,費しては送り仮名が間違えている.

「文書の品質の検査をする」なる文章は,格助詞の「の」が複数回利用されているためエラーとなる.

我輩は猫である.しかし,名前はまだないです.
「です・ます」の混在も指摘される.

このビルでは,1000人が働いている.
2019年には,元号が変わる.
4桁以上の数字には3桁毎のカンマを入れる必要がある.しかしながら,年号の場合には指摘されない設定とした.

\section{空の節}
\subsection{Void Section}
節をたてた場合には,その概要文章が必要である.
節に説明がなく,いきなり小節を書いた場合にはエラーを出力する.

