\documentclass[a4paper,10pt,twocolumn]{jsarticle}
\usepackage{myjlababsstyle}
\begin{document}
\section{基本的な入力方法}
\LaTeX を利用する際に、最初に知っておくべきことは「スペース」や「改行」などが、エディタで入力したとおりにならないことと、キーボード上の記号の中には 「\%」 など、そのまま入力しただけでは出力できない文字が有るということである\footnote{ちなみに\%記号を表示したい場合は、「\verb+\%+」と入力する}。
これらのポイントは、電気通信大学 佐藤研究室による「TeXマニュアル」\cite{HTUlatex}にまとめられている。

以降、特に注意するポイントについてのみ記載する。
\subsection{章と節、節々}
章のタイトルには \verb+\chapter{}+、節のタイトルには \verb+\section{}+を利用する。
また、節の下のレベル(ここでは節々) のタイトルを記載するには \verb+\subsection{}+を利用する。
それぞれ、適切なフォーマットにて番号が付与されて、表示がなされる。

更に下のレベルは、\verb+subsubsection{}+を用いることができる。本スタイルパッケージでは、
このレベルにおいて番号を記載しないようにした。
したがって、このレベルを最小として論文を構成するようにして欲しい。

なお後述の理由から、抄録では \verb+\chapter{}+は利用しない。
。

\subsection{改行と改段落}
\LaTeX では、HTMLを書くときと同様に、エディタ上で「半角空白をいくついれたか」「改行したか」といった情報は、無視される。

改行には 「\verb+\\+」を、改段落には 「\verb+\par+」を利用する\footnote{改段落の場合には「\verb+\par+」を入れるのではなく、空白行を入れる方法を推奨するが、説明として記載している}\footnote{改段落された後の段落は、自動的に一字下げされる。一方で改行の場合には、字下げはなされない。}\footnote{どうしてもスペースを空けたい場合には、「\verb+~+」を利用する。}。

一見不自由に見えるかもしれないが、この特性は論文を書く際に便利な機能である。
まず、論文を書く際に、意図的な改行を入れることはあまりない。
つまり改行の 「\verb+\\+」を使うことは、ほとんど無い。

逆に改段落は、論文を書く際には意識して頻繁に利用する。
ここで、段落が変わる位置に空白行を挿入すると、「\verb+\par+」と入力したことと同じ意味となる。
%したがって、改段落には \verb+\par+を入れるのではなく、空白行を入れる方法を推奨する。

テキストエディタなどで文章を書く際のポイントと効果を以下にまとめる。
\begin{itemize}
\item 一文ずつ改行しながら文章を記述する
\begin{itemize}
\item[・] 行がつながっていない方が、エディタ上の編集では効率的である
\item[・] 改行は、文章の仕上がりの見た目の改行ではない
\end{itemize}
\item 段落が変わる毎に空白行を挿入する
\begin{itemize}
\item[・] エディタ画面では、段落のまとまりがわかりやすい
\item[・] 文章のバランスや量などに気を配ることができる
\end{itemize}
\end{itemize}

\begin{comment}
上記ポイントを実践して記述した本書類の第1章の中身を以下に示す。
\begin{screen}
{\footnotesize
\begin{verbatim}
\section{はじめに}
本論文では、○○○を△△△することにより、□□を明らかとする研究について記述する。

まず、本研究をおこなう背景となった事柄について述べる。
次に、研究目的の詳細を記述した後、類似研究との相違や関連研究とのつながりについて解説する。
また、次章以降の本論文の構成についてその概略を述べる。

\subsection{背景}
研究の目的につながる背景事項を説明する。
その説明には、参考文献やデータを参照するように。

あまり詳しく書きすぎると、2章や3章などで書く内容が無くなったり重複したりしてしまうので、研究の目的の妥当性につながる程度の内容(詳細さ)でかまわない。

\subsection{研究目的}
背景によって、研究の大きな目的が導かれる。
その大きな目的を正確に定義した後、本研究にて実際にターゲットとする目的を詳細に記述する\footnote{大きな目的は1年間の研究ではカバーしきれない為}。

また、背景にて実際の詳細なターゲットの必要性を示した場合には、それの詳細な条件を記載する。
…
\end{verbatim}
}
\end{screen}
\end{comment}
%
\end{document}