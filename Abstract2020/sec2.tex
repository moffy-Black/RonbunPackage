\documentclass[a4paper,10pt,twocolumn]{jsarticle}
\usepackage{myjlababsstyle}
\begin{document}
\section{関連研究}
角田ら(2018)\cite{4}は印象をPlutchikの感情モデルの基本8感情\cite{5}を用いて推定した.形態素解析ツールを用いて歌詞から単語を抽出する.その後Word2Vecを利用してのベクトル変換を行ない,
基本8感情との類似度を算出して,一番類似度が高い感情をテキストから得られる印象とした.この研究では喜び,悲しみなどの推定できた印象と嫌悪,怒りなどの推定できなかった印象が存在した.

大木ら(2018)\cite{6}は日本語評価極性辞書にのっとり,歌詞をポジティブ・ネガティブ・ニュートラルの3つの印象に分類した.3つの評価極性を歌詞内の各単語に付与し,単語の評価極性値を平均化することで歌詞自体への印象の評価値を算出する.
歌詞をポジティブ,ネガティブ,ニュートラルの3印象にうまく推定できるが,より詳細な印象(喜び,悲しみなど)は正確に推定できていない.

上記で述べたとおり,歌詞情報から印象を推定する研究は未だ発展途上であるため,Word2Vecや日本語評価極性辞書以外で印象推定する手法を模索する.
西川らは英語の感情価単語セットANEW\cite{7}を用いて,歌詞をAV平面上で表現する研究を行なった.歌詞の各フレーズが持つ単語のAV平面座標を推定して,フレーズごとの平面座標を求めた.
この印象推定手法は英語の歌詞を用いて効果測定が行われたが,日本語の歌詞での効果測定は行われていないので,日本語の歌詞の場合でもこの手法が効果を発揮するのか調査する.
本間ら\cite{8}はANEWの単語を参考にして,日本語訳した単語セットを作成した.
よって,この日本語の単語セットを利用すれば,西川らが考案した手法を日本語の歌詞の印象推定でも一定の効果を挙げられることが予想される.

%
\end{document}
