\documentclass[a4paper,10pt,twocolumn]{jsarticle}
\usepackage{myjlababsstyle}
\begin{document}
\section{関連研究}
<<<<<<< HEAD
角田ら(2018)\cite{4}は印象をPlutchikの感情モデルの基本8感情\cite{5}を用いて推定した.形態素解析ツールを用いて歌詞から単語を抽出する.その後Word2Vecを利用してのベクトル変換を行ない,
基本8感情との類似度を算出して,一番類似度が高い感情をテキストから得られる印象とした.この研究では喜び,悲しみなどの推定できた印象と嫌悪,怒りなどの推定できなかった印象が存在した.

大木ら(2018)\cite{6}は日本語評価極性辞書にのっとり,歌詞をポジティブ・ネガティブ・ニュートラルの3つの印象に分類した.3つの評価極性を歌詞内の各単語に付与し,単語の評価極性値を平均化することで歌詞自体への印象の評価値を算出する.
歌詞をポジティブ,ネガティブ,ニュートラルの3印象にうまく推定できるが,より詳細な印象(喜び,悲しみなど)は正確に推定できていない.

上記で述べたとおり,歌詞情報から印象を推定する研究は未だ発展途上であるため,Word2Vecや日本語評価極性辞書以外で印象推定する手法を模索する.
西川らは英語の感情価単語セットANEW\cite{7}を用いて,歌詞をAV平面上で表現する研究を行なった.歌詞の各フレーズが持つ単語のAV平面座標を推定して,フレーズごとの平面座標を求めた.
この印象推定手法は英語の歌詞を用いて効果測定が行われたが,日本語の歌詞での効果測定は行われていないので,日本語の歌詞の場合でもこの手法が効果を発揮するのか調査する.
本間ら\cite{8}はANEWの単語を参考にして,日本語訳した単語セットを作成した.
よって,この日本語の単語セットを利用すれば,西川らが考案した手法を日本語の歌詞の印象推定でも一定の効果を挙げられることが予想される.
=======
日本語のテキスト情報から印象を推定する手法についていくつか紹介したのちに,本研究の立ち位置を説明する.

\subsection{日本語テキストから印象を推定する手法}
\subsubsection{対話よるムード推定に基づく楽曲推薦エージェントの開発}
角田ら(2018)[5]は印象をPlutchikの感情モデル[6]の基本8感情を用いて推定した.基本8感情は喜び,信頼,心配,驚き,悲しみ,嫌悪,怒り,予測である.システムとユーザの会話を記録し,会話のテキスト情報をMeCabによる形態素解析をして単語を抽出する.
その後,Word2Vecを利用して抽出した単語のベクトル変換を行ない,基本8感情との類似度を算出して,一番類似度が高い感情をテキストから得られる印象とした.この研究では喜び,悲しみなどの推定できた印象と嫌悪,怒りなどの推定できなかった印象が存在した.


\subsubsection{物体の色や表情情報を利用した画像の印象にあった音楽推薦手法の提案}
追木ら(2018)[7]は印象をAV平面上で表現した.英語テキストをWord2Vecを用いてArousal-Valence値が既知である印象語と類似度を計算した.Arousal-Valence値が既知である印象語はGeorgios(2013)[8]らが公開されしたものを使用している.
テキストともっとも類似度が高い印象語の重みを1として正規化し,各印象の重みつけを行う.各印象語間の角度によるクラスタリング手法であるspherical-kemeans[9]を用いてクラスタリングを行った.クラスタ数は4つであり,AV平面の各象限である.
しかしこの実験では人が認識した印象とシステムが推定した印象の一致率は低かった.

\subsubsection{歌詞解析と心拍変動分析を用いた楽曲によふ感情への影響の予備的調査}
大木ら(2018)[10]は日本語評価極性辞書にのっとり,歌詞をポジティブ・ネガティブ・ニュートラルの3つの印象に分類した.3つの評価極性を歌詞内の各単語に付与し,単語の評価極性値を平均化することで歌詞自体への印象の評価値を算出する.
歌詞をポジティブ,ネガティブ,ニュートラルの3印象にうまく推定できるが,より詳細な印象(喜び,悲しみなど)は正確に推定できない.

\subsection{本研究の位置付け}
上記で述べたとおり,歌詞情報から印象を推定する研究は未だ発展途上である.ユーザの気分に合う楽曲を探すために日本語の歌詞を持つ楽曲がユーザに与える印象を推定する手法を模索する.
西川ら[11]は英語の感情価単語セットANEW[12]を用いて,歌詞をAV平面上で表現する研究を行なった.ANEWとは英単語の1034語について感情価を調査したデータセットである.この研究では歌詞の各フレーズが持つ単語のAV平面座標を推定して,フレーズごとの平面座標を求めた.
この印象推定手法は英語の歌詞での効果測定は行われたが,日本語の歌詞での効果測定は行われていないので,日本語の歌詞の場合でもこの手法が効果を発揮するのか調査する.
本間ら[13]はANEWの単語を参考にして,日本語訳した単語セットを使用し,単語の感情価と覚醒度の評定をする.その結果,AV平面上での単語の分布はBradleyらと同様の分布を得られることが確認された.
よって,本間らが作成した日本語の単語セットを利用すれば,西川らが考案した手法を日本語の歌詞の印象推定でも一定の効果を挙げられることが予想される.
>>>>>>> feature/Abstract

%
\end{document}
