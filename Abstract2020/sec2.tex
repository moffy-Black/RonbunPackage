\documentclass[a4paper,10pt,twocolumn]{jsarticle}
\usepackage{myjlababsstyle}
\begin{document}
\section{宮治研~抄録\LaTeX スタイルパッケージの使い方}
基本的に論文のスタイルパッケージと同様に作業をすれば良い。
たとえば、\verb+main.tex+ファイルに必要事項を記載し、適切なファイルを取り込むように指定し、バッチコマンドを利用すれば、PDFファイルができ上がる。

なお、抄録を記述する際注意事項として、スタイルパッケージの利用方法以外については次節にて解説する。

\subsection{サブタイトル有りの場合}
配布したファイルは、サブタイトルがある場合のサンプルになっている。
まず、年度/学籍番号/氏名、タイトル、サブタイトルを所定の命令内に記入する。
\begin{screen}
{\small
%footnotesize
\begin{verbatim}
\nendo{2019年度}
\snum{15387019}
\jname{宮治 裕}
\thesistitle{宮治研における論文作成について}
\thesissubtitle{\LaTeX の利用}
\end{verbatim}
}
\end{screen}
次に \verb+\SUBTtrue+は命令の先頭に\verb+%+がつかない状態に、\verb+\SUBTfalse+は命令の先頭に\verb+%+がつく状態にする。
\verb+%+ が付いているのは、コメントアウト状態であり、コンパイル処理されないことを示す。
\begin{screen}
{\small
%footnotesize
\begin{verbatim}
\SUBTtrue
%\SUBTfalse
\end{verbatim}
}
\end{screen}

\subsection{サブタイトル無しの場合}
サブタイトル有りの場合と比較して3箇所の変更が必要である。
\begin{enumerate}
\item サブタイトルを記入する命令の先頭部分に \verb+%+ 記号を入れ、コメントアウト状態にする
\item \verb+\SUBTtrue+の前に\verb+%+ 記号を入れ、コメントアウト状態にする
\item コメントアウト状態の \verb+\SUBTfalse+の直前の \verb+%+ 記号を削除する
\end{enumerate}
以上の変更を行った設定を示す。
\begin{screen}
{\small
%footnotesize
\begin{verbatim}
%\thesissubtitle{}
%\SUBTtrue
\SUBTfalse
\end{verbatim}
}
\end{screen}

%
\end{document}